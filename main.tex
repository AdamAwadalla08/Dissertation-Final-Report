\documentclass[12pt]{report}

% PACKAGES:

\usepackage[utf8]{inputenc}
\usepackage{cite}
\usepackage[english]{babel}
\usepackage[a4paper,top=2.5cm,bottom=2.5cm,left=2.5cm,right=2.5cm,marginparwidth=1.75cm,footnotesep=3.5cm]{geometry}
\usepackage{amsmath}
\usepackage{graphicx}
\usepackage{svg}
\usepackage[colorlinks=true, allcolors=black]{hyperref}
\usepackage{amssymb} 
\usepackage{newtxtext}
\usepackage{lmodern}
\usepackage{float}
\usepackage[final]{pdfpages}
\usepackage{caption}
\usepackage{listings}
\usepackage[dvipsnames]{xcolor}
\usepackage{color}
\usepackage{animate}
\usepackage{pgf}
\usepackage{subcaption}
\usepackage{booktabs}
\usepackage{multirow}
% SETTINGS

\setlength{\parindent}{1.5cm}
\captionsetup{font={footnotesize,it}}


\title{

\Large{MEng Mechanical Engineering}\\
\vspace{30pt}
    \huge{\textbf{Developing an Open-Source Frequency Domain Modal Analysis Algorithm in Python}} \\
    \vspace{30pt}
}

\author{
        \Large{Adam AWADALLA} \\
        \vspace{30pt}
}

\date{
        \Large{May 2025} \\
        \vspace{40pt}
        \Large{Timothy J. Rogers} \\
        \vspace{40pt}
        \Large{Report submitted to the University of Sheffield in partial fulfillment of the       
        requirements for the degree of Master of Engineering}\\
        \vspace{10pt}
        Word Count: $\thicksim$ 7500
      }

    
\numberwithin{equation}{chapter}
\numberwithin{figure}{chapter}
\numberwithin{table}{chapter}



% MSVS light theme as it's like the standard for coding I guess ¯\_(ツ)_/¯


\definecolor{keyword}{RGB}{0,0,255}      % Blue for keywords
\definecolor{string}{RGB}{163,21,21}    % Red for strings
\definecolor{comment}{RGB}{0,128,0}     % Green for comments
\definecolor{identifier}{RGB}{0,0,0}    % Black for identifiers
\definecolor{background}{RGB}{255,255,255} % White background
\definecolor{numbering}{RGB}{111,111,111} % Gray for line numbers

% Python code listings

\lstdefinestyle{Python}{
    language=Python,
    basicstyle=\ttfamily\small,              % Font size and family
    keywordstyle=\color{keyword}\bfseries,  % Keywords in blue and bold
    stringstyle=\color{string},             % Strings in red
    commentstyle=\color{comment}\itshape,   % Comments in green and italic
    identifierstyle=\color{identifier},     % Identifiers in black
    backgroundcolor=\color{background},     % White background
    numberstyle=\tiny\color{numbering},     % Gray line numbers
    numbers=left,
    stepnumber=1,
    showstringspaces=false,
    breaklines=true,
    frame=single,                           % Frame around the code
}

\definecolor{githubBg}{RGB}{13,17,23}      % background: 0D1117
\definecolor{githubText}{RGB}{201,209,217}   % text: C9D1D9
\definecolor{githubComment}{RGB}{139,148,158} % comment: 8B949E
\definecolor{githubKeyword}{RGB}{255,128,0}
\definecolor{githubString}{RGB}{165,214,255}  % string: A5D6FF
\definecolor{githubFrame}{RGB}{48,54,61}       % frame: 30363D

\lstdefinelanguage{Python}{
  language=Python,
  morekeywords={def, class, import, from, as, if, else, elif, for, while, return, print},
  literate=
  {"""}{{\lstinline!"\"\"\"!}}3
  {'''}{{\lstinline!\'\'\'!}}3,
  sensitive=true,
  comment=[l]\#,
  string=[b]",
  string=[b]',
  stringstyle=\color{teal},
  commentstyle=\itshape\color{gray},
}

\lstdefinestyle{github-dark}{
  language=Python,
  backgroundcolor=\color{githubBg},
  basicstyle=\ttfamily\footnotesize\color{githubText},
  commentstyle=\color{githubComment}\itshape,
  keywordstyle=\color{githubKeyword}\bfseries,
  stringstyle=\color{githubString},
  identifierstyle=\color{githubText},
  frame=single,
  rulecolor=\color{githubFrame},
  numbers=left,
  numberstyle=\tiny\color{githubFrame},
  showstringspaces=false,
  breaklines=true,
  captionpos=b,
  tabsize=2,
}


\definecolor{qlBackground}{HTML}{F6F6F6}
\definecolor{qlForeground}{HTML}{333333}
\definecolor{qlComment}{HTML}{6A9955}
\definecolor{qlKeyword}{HTML}{0000FF}
\definecolor{qlString}{HTML}{A31515}
\definecolor{qlIdentifier}{HTML}{001080}
\definecolor{qlNumber}{HTML}{098658}
\definecolor{qlType}{HTML}{267f99}
\definecolor{qlPunctuation}{HTML}{393A34}

\lstdefinestyle{quietlight}{
  backgroundcolor=\color{qlBackground},
  basicstyle=\ttfamily\small\color{qlForeground},
  commentstyle=\color{qlComment}\itshape,
  keywordstyle=\color{qlKeyword}\bfseries,
  stringstyle=\color{qlString},
  identifierstyle=\color{qlIdentifier},
  numberstyle=\color{qlNumber},
  emphstyle=\color{qlType},
  moredelim=[s][\color{qlPunctuation}]{\{}\{,   % punctuation around braces
  moredelim=[s][\color{qlPunctuation}]{\}}{\,},
  showstringspaces=false,
  numbers=left,
  numberstyle=\tiny\color{gray},
  numbersep=5pt,
  breaklines=true,
  frame=single,
  rulecolor=\color{gray!50},
  tabsize=2,
  captionpos=b,
  upquote=true
}

\lstset{%
  basicstyle=\ttfamily,
  literate={"}{{\ttfamily\char`\"}}1
}

\lstdefinelanguage{Pseudo}{
  keywords={FUNCTION, RETURN, FOR, IF, THEN, END, ENDIF, ENDFOR},
  sensitive=true,
  morecomment=[l]{\#},
  morestring=[b]"
}
\lstdefinestyle{pseudocode}{
  language=Pseudo,
  frame=single,
  backgroundcolor=\color{gray!10},
  basicstyle=\ttfamily\small,
  keywordstyle=\color{blue}\bfseries,
  commentstyle=\itshape\color{gray},
  stringstyle=\color{teal},
  numbers=left,
  numberstyle=\tiny\color{gray},
  stepnumber=1,
  numbersep=8pt,
  tabsize=2,
  showstringspaces=false,
  breaklines=true,
  captionpos=b
}

\newcounter{nalg}[chapter] % defines algorithm counter for chapter-level
\renewcommand{\thenalg}{\thechapter .\arabic{nalg}} %defines appearance of the algorithm counter
\DeclareCaptionLabelFormat{algocaption}{Algorithm \thenalg} % defines a new caption label as Algorithm x.y

\lstnewenvironment{algorithm}[1][] %defines the algorithm listing environment
{   
    \refstepcounter{nalg} %increments algorithm number
    \captionsetup{labelformat=algocaption,labelsep=colon} %defines the caption setup for: it ises label format as the declared caption label above and makes label and caption text to be separated by a ':'
    \lstset{ %this is the stype
        mathescape=true,
        frame=tB,
        numbers=left, 
        numberstyle=\tiny,
        basicstyle=\scriptsize, 
        keywordstyle=\color{black}\bfseries\em,
        keywords={,input, output, return, datatype, function, in, if, else, foreach, while, begin, end, } %add the keywords you want, or load a language as Rubens explains in his comment above.
        numbers=left,
        xleftmargin=.04\textwidth,
        #1 % this is to add specific settings to an usage of this environment (for instnce, the caption and referable label)
    }
}
{}



\setlength{\fboxsep}{1pt}   

% \newcommand{\inlinecode}[1]{%
%   \colorbox{gray!10}{\ttfamily\small #1}%
% }

% or, if you’d also like a thin border:
\newcommand{\inlinecode}[1]{%
  \fcolorbox{gray!50}{gray!10}{\ttfamily\small #1}%
}


\begin{document}
\numberwithin{lstlisting}{chapter}

\pagenumbering{gobble}
\setcounter{page}{1}
\pagestyle{plain}
\begin{figure}
    \centering
    \includegraphics[width=\linewidth]{figures/unilogoplusmecheng.png}
\end{figure}
\maketitle

\tableofcontents

\pagebreak
\chapter*{Acknowledgements}
\addcontentsline{toc}{chapter}{Acknowledgements}
I would like to express my deepest gratitudes to my primary supervisor, Dr. Tim Rogers, and my secondary supervisors, Dr. Max Champneys and Dr. Brandon O'Connel, for all of their unwavering support, guidance, and understanding throughout this project. Thanks also to my partner, Katie,  without whom I would not have submitted this work. Finally, my Father, for supporting me, his financial burden.

To everyone mentioned here, and to all those who have contributed—whether through discussion, critique, or moral support—I offer my heartfelt thanks.
\clearpage

\chapter*{Abstract}
\addcontentsline{toc}{chapter}{Abstract}
Structural dynamics relies on modal analysis to characterise vibratory behaviour in engineering systems, yet commercial software obscures underlying algorithms and imposes high licensing costs. This work aims to deliver a transparent, open-source alternative by implementing the polyreference least-squares complex frequency-domain estimator (pLSCF, PolyMAX) and a least-squares frequency-domain (LSFD) mode-shape routine in Python. Methods include a NumPy/SciPy - based architecture, rigorous unit- and integration-testing, PEP-8 documentation, and model-order-reduction via matrix truncation and singular-value decomposition to accelerate computation. Results on a ten-degree-of-freedom benchmark show natural frequencies and damping ratios within $9 \times 10^{-5} \%$  of analytical values; mode shapes reach modal-assurance-criterion scores $\geq$ 0.997. Computational time is reduced by up to two orders of magnitude relative to brute-force fitting, and accuracy is retained under 25 dB signal-to-noise conditions after peak-segmentation pre-processing. The study concludes that the presented library supplies a robust, high-performance, and freely accessible alternative to proprietary tools while providing a foundation for future enhancements such as MAC-based stability criteria, PolyMAX Plus noise handling, and an interactive graphical interface.
\clearpage



\pagenumbering{arabic}
\setcounter{page}{1}
\pagebreak
\section{Introduction}\label{sec:intro}
The theoretical and experimental study of structural dynamics has regularly helped engineers grasp the behaviour of systems and structures encountered in everyday life, and has consequently aided in making them safer, lighter, and greener. This ranges from the design of aeroplanes and cars to the stability of buildings as well as the functionality of household appliances like washing machines or air conditioners. Hence, engineers and researchers have been consistently striving to further understand and predict the dynamic characteristics of these various structures. This understanding is crucial for ensuring safety and compliance, performance, and reliability. \cite{Rao2018,Maia2024,Worden2001nonlinear}

\subsection{Modal Analysis}\label{sec:intro-modal}
\subsubsection{Theoretical Overview}\label{sec:intro-modal-maths}
In linear structural dynamics, Modal Analysis is universally recognized as the pre-eminent solution for the identifying and characterizing structures or systems. It achieves this by studying the structure's modal properties or parameters — its natural frequencies, mode shapes, and damping ratios. Whether using mathematical modelling or experimental testing, modal analysis is key for engineers to understand the response of structures to various excitations or forces, ensuring safety, compliance with standards and regulations, and supporting many research areas such as Structural Health Monitoring or System Identification. \cite{ewins2000modal,fu2001modal,Maia1997}

Fundamentally, modal analysis is the decomposition of the complex oscillatory behaviour of structures into smaller, more intuitive components called modes. Each mode is a mathematical representation of a specific vibration pattern associated with a natural frequency, the frequency (or set of) at which a system tends to oscillate when displaced, and a corresponding mode shape, a vector which describes the relative movement among the degrees-of-freedom. The damping ratio, a unitless parameter, quantifies the energy dissipation of the system for each mode. The modal properties are defined by the interaction of the system's physical properties, its mass, stiffness, and damping. These inherent properties guide the system's vibration when subject to external forcing or initial displacements or velocities.

The standard procedure for modal analysis begins with forming the equations of motion (EOMs) that represent a system, typically, second order matrix differential equations. Consider a system with an arbitrary number $N$, degrees-of-freedom (DOFs) as shown in Figure \ref{fig:ndof-chain}. Using Newton's second law \cite{thornton2014classical},
 $$\sum F_i = m_i\ddot{\mathbf{x}}_i$$ 
 Where $F_i$ is a force acting on the $i$-th DOF, $m_i$ is the mass, and $\mathbf{x}_i$ is the coordinate, or alternatively the Euler-Lagrange equation as such \cite{thornton2014classical}: 
 $$\frac{d}{dt}\left(\frac{\partial T}{\partial \dot{q}_i}\right) - \frac{\partial T}{\partial q_i} + \frac{\partial U}{\partial q_i} = Q_i$$
Where $q_i$ is a generalized coordinate (take $q_i = x_i$ for this system), $T$ is the system's kinetic energy, $U$ is the potential energy, and $Q_i$ represents the non-conservative forces. The resulting equations of motion for the system are given in Equation \ref{eq:ndof-eoms}.

\begin{equation}\label{eq:ndof-eoms}
    \begin{aligned}
        m_1\ddot{x}_1 + (c_1+c_2)\dot{x}_1 - c_2\dot{x}_2 + (k_1+k_2)x_1 -k_2x_2 & = F_1 \\
        m_2\ddot{x}_2 + (c_2+c_3)\dot{x}_1 - c_2\dot{x}_1 - c_3\dot{x_3} + (k_2+k_3)x_2 -k_2x_1 -k_3x_3 & = F_2\\
        \dots \hspace{5cm}\\
        m_N\ddot{x}_N + (c_N+c_{N+1})\dot{x}_N - c_{N-1}\dot{x}_{N-1} + (k_N+k_{N+1})x_N -k_{N-1}x_{N-1} & = F_{N}
    \end{aligned}
\end{equation}

\begin{figure}[H]
    \centering
    \includegraphics[width=\linewidth]{Figures/n-dof-chain-sys.png}
    \caption{A "lumped-mass" system of $N$ degrees-of-freedom}
    \label{fig:ndof-chain}
\end{figure}

By assembling the physical parameters into respective matrices as highlighted in Equation \ref{eq:system-matrices}, the characteristic differential equation of the system is obtained:

\begin{equation}\label{eq:system-matrices}
    \begin{aligned}
        \relax [\mathbf{K}] &= \begin{bmatrix}
            k_1+k_2&-k_2 & 0 &\dots & 0\\
            -k_2&k_2+k_3&-k3&\dots &0\\
            0 & -k_3 & k_3+k_4& -k_4 &0\\
            \dots&\dots&\dots&\ddots &\dots\\
            0 & 0 & 0 & -k_{N} & k_{N}+k_{N+1}
        \end{bmatrix}\\
        [\mathbf{C}] &= \begin{bmatrix}
            c_1+c_2&-c_2 & 0 &\dots & 0\\
            -c_2&c_2+c_3&-c3&\dots &0\\
            0 & -c_3 & c_3+c_4& -c_4 &0\\
            \dots&\dots&\dots&\ddots &\dots\\
            0 & 0 & 0 & -c_{N} & c_{N}+c_{N+1}
        \end{bmatrix}\\
        [\mathbf{M}] &= \mathbf{diag}(m_i) \hspace{1cm} \forall i = 1,2,\dots,N
    \end{aligned}
\end{equation}
\begin{equation}\label{eq:general_damped_mdof}
    \therefore [\mathbf{M}]\mathbf{\ddot{x}}+[\mathbf{C}]\mathbf{\dot{x}}+[\mathbf{K}]\mathbf{x} = \mathbf{F}
\end{equation}

For simplicity, assume the system presented vibrates freely, and is undamped, so $\mathbf{F}$ and $[\mathbf{C}]$ are zero. If the solution of the presented differential equation is harmonic, i.e. $\mathbf{x}(t) = \Psi e^{j\omega t}$, one finds that equation \ref{eq:general_damped_mdof} reduces to:
\begin{equation}
    ([\mathbf{K}] - \omega^2[\mathbf{M}])\Psi = 0
\end{equation}
Rearranging the terms, the equation into takes the general form of an eigenvalue problem:
\begin{equation}\label{eq:eigen-solution}
    \begin{aligned}
        \relax [\mathbf{A}]& =  [\mathbf{M}]^{-1}[\mathbf{K}]\\
        [\mathbf{A}]&\Psi = \omega^2\Psi
    \end{aligned}
\end{equation}
Here, $\omega^2$ is the eigenvalue of matrix $\mathbf{A}$  — often referred to as an \emph{eigenfrequency} — and $\Psi$ is the corresponding eigenvector representing the mode shape.

% In systems with a finite number of degrees of freedom—often referred to as discrete systems—the modal analysis is performed using the dynamical matrix $[\mathbf{A}]$, which encapsulates the system's inertial and elastic properties. Building upon the discrete case, into systems which are modelled as continuous functions, the physical parameters of the systems are distributed spatially. This results into partial differential equation which governs the motion, where  finite dimensional matrices are then replaced with linear operators. Consequently, an eigenvalue problem is formed differently, in terms of a differential operator, where the eigenvalues are still correspondent to the natural frequency, and the respective eigenfunctions are the modeshapes in the spacial domain.

In discrete systems, with a finite number of degrees of freedom, the modal analysis is performed using the matrix $[\mathbf{A}]$, which encapsulates the system's inertial and elastic properties. When extending the approach to systems modelled as continuous mediums, the physical parameters are distributed spatially. This results in a partial differential equation that governs the motion, where finite-dimensional matrices are replaced by linear operators. Consequently, an eigenvalue problem is formed in terms of a differential operator, with eigenvalues corresponding to the natural frequencies and the associated eigenfunctions representing the mode shapes in the spatial domain\footnote{An \emph{eigenvector}, or an \emph{eigenfunction}, of a matrix or linear operator defined on some vector/function space is any non-zero vector/function in said space that when multiplied or acted upon by the matrix/linear operator is equivalent to being multiplied by some scalar factor, said scalar factor is referred to as the \emph{eigenvalue}.}. For example this principle applied to transverse vibration of an Euler-Bernoulli beam yields the eigenvalue problem:
\begin{equation}\label{eq:eigenfunction-beam}
    \Delta^2Y(x) = \beta^4Y(x)
\end{equation}
where $\Delta$ is the Laplacian operator (i.e. second partial derivative in Cartesian coordinate system), and $\beta^4 = \frac{\omega^2}{c^2}$, in which $\omega$ is the natural frequency, and $c^2$ is the ratio of the beam's flexural rigidity and its inertia. \cite{Rao2018,Blevins2015}

In both discrete and continuous cases, the process of reducing the governing equations into eigenvalue problems is mathematically elegant, robust, and applicable to all linear systems. Despite this approach's elegance, its practical applications rarely exist, as real world structures and systems seldom conform to the idealized assumptions of lumped-mass systems or Euler-Bernoulli beams with known boundary conditions. Thus, academics and practitioners are driven to approach modal analysis differently, to compensate for this limitation.

\subsubsection{Modal Analysis in practice, academia, and industry}\label{sec:intro-modal-applications}

There are two mainstream modal analysis procedures which address this limitation: Numerical Modal Analysis, and Experimental Modal Analysis, often referred to as Modal Testing. Numerical Modal Analysis is used to simulate dynamic behaviour when modal testing is unimplementable or unnecessary. It involves discretization — typically using finite element modelling — and solving the resulting equations of motion under appropriate initial and boundary conditions in a process similar to that shown in Equations \ref{eq:ndof-eoms}-\ref{eq:eigen-solution}. Numerical Modal Analysis offers a faster means of evaluating simpler systems without the need for experimental testing. 

In contrast, Modal Testing relies on the principle that, in a mostly linear system, the same modal parameters used to predict the system's response can be obtained from a measurement of that response. It utilizes various experimental methods such as shaker testing or impact testing to gather response data using sensors like accelerometers or laser vibrometers. Typically, one would pair the experimental set-up with computational algorithms to extract the modal parameters from the test data. An example set-up for a modal test using a shaker is showcased in Figure \ref{fig:shaker-test}. Modal Testing is often the first step for applications such validating models, or measuring the effect of environmental and operational conditions on structures. This underscores why modal testing is typically preferred and more prevalent, as it directly captures the system's response in a way that numerical modal analysis is unable to. Due to the widespread adoption of modal testing in structural dynamics, the term modal analysis is generally interpreted as modal testing, and due to the nature of the work in this report, the term modal analysis will also be referring to modal testing.

\begin{figure}[H]
    \centering
    \includegraphics[width = 0.75\linewidth]{Figures/Shaker-test.png}
    \caption{Typical electrodynamic shaker experimental set up, adapted from He \& Fu \cite{fu2001modal}}
    \label{fig:shaker-test}    
\end{figure}

In modern research areas such as structural health monitoring (SHM)—where the primary objective is to observe changes in an asset's structural condition through continuous monitoring—modal analysis is frequently employed because a system's modal parameters are highly sensitive to its physical state \cite{WordenSHM2013}. Natural frequencies and mode shapes are considered damage-sensitive features; whether changes result from cracks, corrosion, environmental effects, or fatigue, any structural alteration will modify the physical parameters and lead to a quantifiable change in the modal parameters. By statistically comparing a structure's current modal parameters with its baseline measurements, typically taken when the structure is in good condition, early signs of damage and degradation can be detected. This approach enables the effective monitoring of essential infrastructure, including bridges \cite{Bunce2024,Maeck2001,Peeters2000}, buildings \cite{Saeed2024}, and wind turbines \cite{Bull2021,Tsiapoki2024}, making modal analysis an invaluable tool for ensuring the safety and longevity of many everyday structures, Making modal analysis an invaluable tool for ensuring the safety and longevity of many everyday structures. See Figure \ref{fig:z24-bridge}, and Figure \ref{fig:hawk-tuah} for examples of modal testing in SHM contexts for civil and structural engineering and for aerospace respectively.
\begin{figure}[h]
    \centering
    \includegraphics[width = 0.75\linewidth]{Figures/z24 modeshapes color.png}
    \caption{Model analysis results of the Z24 bridge where Maeck et al. in \cite{Maeck2001,MAECK2003,Peeters2000} performed an experimental campaign on a bridge in Switzerland to benchmark vibration based techniques for damage identification}
    \label{fig:z24-bridge}
\end{figure}

Modal analysis is widely used across various engineering fields. In the automotive industry, it plays an essential role in enhancing vehicle design, safety, comfort, and performance by analysing responses to factors like axle gear noise, unbalanced loads, and road harshness, thereby optimizing lightweight and high-strength designs \cite{Glen2003nvh,Nissan1975,Martz1979,French1998,Liu2020}. In the aerospace sector, detailed dynamic analysis is crucial for balancing structural integrity with weight reduction. Here, modal analysis verifies computational models, controls unwanted vibrations, and evaluates effects such as wind-induced loads and fluid-induced phenomena. Together, these applications provide vital insights that improve overall structural resilience and performance \cite{fu2001modal,ewins2000modal,wright2008aeroelastics,Chamberlain2017,saffry2014,F16gvt2011,UAV2012,Alexander2024}.
\begin{figure}[h]
    \centering
    \includegraphics[width=0.75\linewidth]{Figures/F16 test.png}
    \caption{Peeters et al. \cite{F16gvt2011}, performed modal tests on wind-tunnel model and full sized F16 aircraft}
    \label{fig:f16-test}
\end{figure}

\begin{figure}[h]
    \centering
    \includegraphics[width=0.75\linewidth]{Figures/hawk-tuah.jpg}
    \caption{BAE systems Hawk T1A aircraft}
    \label{fig:hawk-tuah}
\end{figure}

In civil and structural engineering, modal analysis — particularly Operational Modal Analysis (OMA) — is extensively used to predict and evaluate the behaviour of structures subjected to seismic, crowd, wind, wave, or traffic-induced loading. This technique is critical for assessing tall buildings, dams, and bridges, where accurate response measurements and parameter identification directly inform design decisions that enhance resilience and safety. Ultimately, the insights provided by modal analysis help extend the lifespan of structures and safeguard human lives. \cite{fu2001modal,ewins2000modal, Rainieri2014,Archila2013,Abdelnour2024,Saeed2024,Reynolds2009,Roia2015,Cantieni2004,Peeters2000,Maeck2001,Bunce2024}

\pagebreak
\subsection{Computation in engineering, and the software engineering industry}

Building on the earlier discussion that highlighted the integration of computation with modal testing, it is evident that the engineering industry is heavily reliant on software and computational methods. This reliance arises from the extensive volumes of data—such as laboratory results or continuous monitoring outputs—that require processing, as well as from the complexity of calculations that are impractical to perform manually. Common examples include the use of advanced 3D modelling and drawing software, such as Fusion and SolidWorks, programming for data analysis using languages like Python, MATLAB or C++, and various simulation tools like Simulink, ANSYS, or OpenFOAM. It is a reasonable assumption that software usage is indispensable to engineering practices.

Multiple classifications, interpretations, and naming conventions exist among software developers regarding software categorization. To avoid ambiguity in this report, the terms "proprietary software" and "open-source software" are defined explicitly. Open-source software (OSS) refers to software licensed to allow free use, modification, and redistribution of both the software and its source code. In contrast, proprietary software denotes software tools or programming environments that require the purchase of a licence for use and restrict access to the source code, thereby preventing modification and redistribution. Previous work \cite{InterimReport} presented a more comprehensive comparison of the two software distribution paradigms. To avoid redundancy, only a brief summary of those findings is provided here, followed by an overview of the conclusions and the implications for future work.

The earliest efforts in software commercialization demonstrated considerable foresight, causing a widespread adoption of proprietary tools in engineering industries. These tools gained traction due to rigorous testing, continuous support, adherence to industry standards, and user-friendly interfaces that reduced errors and enhanced usability. Conversely, inherent drawbacks such as high costs, usage restrictions, lack of source code access, potential obsolescence, and privacy concerns have emerged. In contrast, open-source alternatives offer cost-free access, greater user control, and community-driven support, though they often face challenges with limited testing and installation complexity. As a result, selecting between open-source and proprietary solutions requires a careful evaluation of these trade-offs, balancing reliability and robustness against cost and flexibility.

\subsection{Project Scope}\label{sec:intro-scope}
The pivotal role that modal analysis plays in advancing engineering fields provides motivation for this project. As highlighted in Section \ref{sec:intro-modal-applications}, modal analysis is essential for ensuring the comfort, performance, and safety of vehicles and aircraft, as well as for safeguarding critical infrastructure such as buildings, bridges, railway tracks, and offshore structures. Despite its widespread application, many engineers and researchers rely on software tools to perform modal analysis without fully understanding the underlying theory and algorithms, which hinders their ability to interpret results or adapt the methods for unique challenges that are presented quite frequently. This reliance has enabled companies, e.g. Siemens or Structural Vibration Solutions, to charge exorbitant prices for software licences — a situation that thrives in a market dominated by a monopoly of robust solutions. To address this gap, the project introduces an accessible, well-documented, and rigorously tested open-source software tool that produces accurate results, elucidates the fundamental theory, and permits customization for novel applications. The development of a Python library for modal analysis represents the first robust and reliable open-source solution in this domain.
\subsubsection{Defining a clear Aim and realistic Objectives}\label{sec:intro-scope-aimsobjs}
This project is part of a broader development initiative aimed at developing and providing modal-analysis open-source software, making it essential to establish a clear aim that outlines specific objectives and goals. Accordingly, this project contributes to the initiative by integrating and testing the polyreference least-squares complex frequency domain modal parameter estimator (pLSCF, or PolyMAX{\copyright} commercially) in a Python library. Modal Analysis assists in engineering design decisions for compliance standards, and various research contexts, such as structural health/condition monitoring, and system identification. The software aims reduce uncertainty in data interpretation, allowing engineers and researchers to direct their focus on efficient experimental design and testing. This aim can be effectively achieved through the following objectives, which provide a structured approach to achieving a full implementation of pLSCF in the library. 

\begin{itemize}
    \item Research the pLSCF method, and obtain a complete algorithmic understanding of the method for implementation.
    \item Integrate pLSCF in the open-source Python library.
    \item Perform unit tests on the algorithm's individual functions and classes, and test the complete algorithm using simulated and experimental data as a benchmark for the algorithm's efficiency and consistency.
    \item Perform the same objectives for the Least-Squares Frequency Domain method, which is a simpler, supplementary algorithm used to estimate the mode shapes.
    \item Help in further supplementary tasks, like signal processing and conditioning, algorithm optimization, and providing documentation for users, which will contribute to the usability and robustness of the published software.
\end{itemize}

% \pagebreak
% Report Story:\\
% \textbf{Structural Dynamics in engineering}
% \begin{itemize}
%     \item How studying structural mechanics has helped us have safer, lighter, greener structures.
%     \item Modal Analysis, is regarded as \textbf{the} solution for linear structural dynamics.
%     \item Maths of modal analysis:
%     \begin{itemize}
%         \item EOM formulation.
%         \item Eigendecomposition of state matrices.
%         \item FRF in modal terms. (refer to plscf solution using that)
%     \end{itemize}
%     \item This is in various industries, home appliances, aero, civil/structural, acoustics etc.
%     \item Experimental Modal Analysis king.
%     \item Curve-fitting methods for modal analysis.
%     \item The need for algorithms in practice.
% \end{itemize}

% Onto software usage in engineering contexts
% \begin{itemize}
%     \item engineers consistently rely on software tools, this is great as it streamlines the important processes.
%     \item foss vs. prop. discuss why it's cheaper in long run, and better for everyone involved.
%     \item Revisit the aims and objectives.
% \end{itemize}


\section{Algorithm Development}\label{sec:methods}
\textbf{maths-y bit}

\begin{itemize}
    \item Frequency response function
    \item dynamical system poles
    \item logic/control flow
    \item complexities and big O notation
    \item unit testing and integration testing.
    \item version control.
\end{itemize}


\begin{itemize}
    \item lsce (brief)
    \item lscf (brief)
    \item plscf (important bits and refer to appendix for full derivation, use own notation and wording).
    \item Why the companion matrix solution works. $\mathbf{eig}(C(p)) = \lambda_i \rightarrow p(\lambda) = 0$
    \item here one must explain what poles are for a system.
    \item lsfd for modeshapes.
\end{itemize}

\textbf{software-y bit}
\begin{itemize}
    \item time complexity optimization. discuss that O notation doesn't always equal less time.
    \item unit testing, explain pytest stuff and fixturing and bla bla bla.
    \item using numpy (BLAS routines/subroutines)
    \item user facing code.
    \item formatting guidelines.
\end{itemize}



\section{Results}

\begin{itemize}
    \item plscf on simulated modal data.
    \begin{itemize}
        \item clean
        \item noisy
        \item slightly nonlinear data (mimo where $H_{ij} \approx H_{ji}$ but $H_{ij} \neq H_{ji}$)
        \item high dofs.
    \end{itemize}
    \item plscf on actual lab data.
    \item adam's plscf vs siemens lms polymax on same dataset.
    \item interpretation of stabilization diagram.
\end{itemize}


\begin{lstlisting}[style={Python}]
import numpy as np

def _make_polynomial_basis_fcn(
    polynomial_order: int, frequency_vector: np.ndarray, sampling_frequency: float
):
    """Function that creates a Polynomial Basis Function matrix

    Args:
        polynomial_order (int): Order of the polynomial created. i.e. if 2 then P(x) = a0 * x^0  +  a1 * x^1 + a2 * x^2
        in the case of pLSCF, Omega(w) = P(e^{jw _delta t})

        frequency_vector (np.ndarray): vector of frequencies measured or simulated, can be hz or rads-1.
        MUST be either a row or column vector/1D Array

    Returns:
        Polynomial basis function matrix.
    """
    # the polynomial basis function is actually the vandermonde matrix for the polynomials, A and B.
    dt = 1 / sampling_frequency  # Sampling rate
    s = np.exp(1.j*frequency_vector*dt) # this is the "x" in the polynomial
    return np.vander(x=s,N=polynomial_order+1,increasing=True)    
\end{lstlisting}

\pagebreak





Consider the Jacobian matrix, for a system with $n_{outputs} = 3$:
\begin{equation}
    J = \begin{pmatrix}
        X_1 & 0 & 0 & Y_1\\
        0 & X_2& 0 & Y_2\\
        0 & 0 & X_3 & Y_3
    \end{pmatrix}    
\end{equation}

\begin{equation}
    J^HJ = \begin{pmatrix}
        X_{1}^HX_1 & 0 & 0 & X_{1}^HY_1\\
        0 & X_{2}^HX_2 & 0 & X_{1}^HY_2\\
        0 & 0 & X_{2}^HX_2 & X_{1}^HY_3\\
        X_{1}^*Y_{1}^T & X_{2}^*Y_{2}^T & X_{3}^*Y_{3}^T & Y_{1}^HY_1+Y_{2}^HY_2+Y_{3}^HY_3\\
    \end{pmatrix}
\end{equation}
if:
$$    R_o = Re(X_{o}^HX_o) $$
$$    S_o = Re(X_{o}^HY_o)$$
$$    T_o = Re(Y_{o}^HY_o)$$
\begin{equation}
    2Re(J^HJ)\theta = 2Re\begin{pmatrix}
        X_{1}^HX_1 & 0 & 0 & X_{1}^HY_1\\
        0 & X_{2}^HX_2 & 0 & X_{1}^HY_2\\
        0 & 0 & X_{2}^HX_2 & X_{1}^HY_3\\
        X_{1}^*Y_{1}^T & X_{2}^*Y_{2}^T & X_{3}^*Y_{3}^T & Y_{1}^HY_1+Y_{2}^HY_2+Y_{3}^HY_3\\
    \end{pmatrix}
    \begin{pmatrix}
        \beta_1\\
        \beta_2\\
        \beta_3\\
        \alpha
    \end{pmatrix} 
\end{equation}
\begin{equation}
    = 
    2 \begin{pmatrix}
        R_1\beta_1+S_1\alpha\\
        R_2\beta_2+S_2\alpha\\
        R_3\beta_3+S_3\alpha\\
        S_{1}^T\beta_1+S_{2}^T\beta_2+S_{3}^T\beta_3 +(T_1+T_2+T_3)\alpha 
    \end{pmatrix}
\end{equation}

which corresponds to the solutions in the normal equations in terms of alpha and beta.


\begin{equation}
    l^{LS}(\theta) = tr\{\theta^TRe(J^HJ)\theta\}
\end{equation}

from vector calculus, if
\begin{equation}
    \mathbf{f}= \mathbf{x^TBx}
\end{equation}
then,
\begin{equation}
    \frac{\partial \mathbf{f}}{\partial \mathbf{x}} = 2\mathbf{Bx}
\end{equation}

then the normal equation for polymax should be:
\begin{equation}
    \frac{\partial l^{LS}}{\partial \theta} = 2Re(J^HJ)\theta
\end{equation}


\pagebreak
\appendix
\chapter{Multi-degree-of-freedom systems}

\section{Forced Vibration Solution}\label{sec:damped-mdof-frf}
\label{sec:frfappendix}
The following is the author's interpretation and summary of the work by Fu in \cite{fu2001modal}, Ewins in \cite{ewins2000modal}, and Rogers in \cite{Tim_FRF} and \cite{Tim_Modal_Properties2}
As previously highlighted in Section \ref{sec:intro}, the solution to a dynamic system can be represented by an eigenvalue problem:
\begin{equation}
    (\textbf{K}-\omega^{2}\textbf{M})\{\varphi\} = 0
    \label{eigen_new}
\end{equation}
Considering 2 modes, $i$ and $j$, equation \ref{eigen_new} becomes:

\begin{equation}
    (\textbf{K}-\omega^{2}_{i}\textbf{M})\{\varphi\}_{i} = 0
    \label{mode_i}
\end{equation}
\begin{center}
    and
\end{center}

\begin{equation}
    (\textbf{K}-\omega^{2}_{j}\textbf{M})\{\varphi\}_{j} = 0
    \label{mode_j}
\end{equation}

Pre-multiplying \ref{mode_i} by $\{\varphi\}_{j}^{T} $, and Transposing \ref{mode_j} and post-multiplying by $\{\varphi\}_{i}$ results in:
\begin{equation}
    \{\varphi\}_{j}^{T}(\textbf{K}-\omega^{2}_{i}\textbf{M})\{\varphi\}_{i} = 0
    \label{zeby1}
\end{equation}
\begin{center}
    and
\end{center}
\begin{equation}
    \{\varphi\}_{j}^{T}(\textbf{K}-\omega^{2}_{j}\textbf{M})\{\varphi\}_{i} = 0
    \label{zeby2}
\end{equation}
subtracting  \ref{zeby1} and \ref{zeby2}:

\begin{equation}
    (\omega^{2}_{i} -\omega^{2}_{j})\{\varphi\}_{j}^{T} \textbf{M} \{\varphi\}_{i} = 0
    \label{zeby3}
\end{equation}
suggesting that for $i \neq j$,  $\{\varphi\}_{j}^{T} \textbf{M} \{\varphi\}_{i} = 0$, substituting this property in equation \ref{zeby2}, $\{\varphi\}_{j}^{T} \textbf{K} \{\varphi\}_{i} = 0$ holds true.
Considering the case where $i = j$, let 
\begin{equation}
    \{\varphi\}_{i}^{T} \textbf{M} \{\varphi\}_{i} = m_{i} 
    \label{massmodal} 
\end{equation}
\begin{center}
    and
\end{center}
\begin{equation} 
    \{\varphi\}_{i}^{T} \textbf{K} \{\varphi\}_{i} = k_{i} 
    \label{modal_stiffness} 
\end{equation}
where $m_{i}$ and $k_{i}$ are the modal mass and stiffness respectively, the eigenvalue for each mode $\lambda_{i}$ can be represented as:
$$\lambda_{i} = \frac{k_{i}}{m_{i}}$$.
The properties highlighted by  equations \ref{zeby1}, \ref{zeby2}, \ref{zeby3}, \ref{massmodal} and \ref{modal_stiffness} are expressed in matrix form more concisely, where:
\begin{equation}
    [\Phi]^{T}\textbf{M} [\Phi] = \text{diag}(m_{i})
    \label{modal_matrix_mass}
\end{equation}
\begin{center}
    and
\end{center}
\begin{equation}
    [\Phi]^{T}\textbf{K} [\Phi] = \text{diag}(k_{i})
    \label{modal_matrix_stiff}
\end{equation}
Referring to the mathematical representation of the frequency response function, from equation \ref{FRF_general_form}, where $H(\omega) =  (\textbf{K}-\omega^{2}\textbf{M})^{-1}$, 
one can express the FRF of a system in terms of its modal parameters as follows,
\begin{equation}
    [\Phi]^{T} (\textbf{K}-\omega^{2}\textbf{M})[\Phi] =  [\Phi]^{T} [H(\omega)]^{-1} [\Phi]
\end{equation}
Substituting from \ref{modal_matrix_mass} and \ref{modal_matrix_stiff}, the FRF matrix is expressed as:
\begin{equation}
    [H(\omega)] = [\Phi][\omega_{i}^{2}-\omega^{2}][\Phi]^{T}
\end{equation}
For a single element in a receptance FRF matrix $H_{oj}(\omega)$, the equation representing it is:
\begin{equation}
    H_{jk}(\omega) = \frac{\varphi_{j1}\varphi_{k1}}{\omega_{1}^{2}-\omega^{2}} + \frac{\varphi_{j2}\varphi_{k2}}{\omega_{2}^{2}-\omega^{2}} + \dots + \frac{\varphi_{jn}\varphi_{kn}}{\omega_{n}^{2}-\omega^{2}}
    \label{FRF_MODAL}
\end{equation}
where $n$ is the number of modes. Equation \ref{FRF_MODAL} indicates that the value of the FRF at a single frequency line can be interpreted as 
the contribution of all individual modes to the specific vibration pattern associated with the frequency line. Essentially the FRF is represented as a sum of partial fractions.
This representation of the FRF is the basis of the validity of the pLSCF method, 
as it similarly assumes a rational fractional representation of the FRF.


\chapter{Derivation of p-LSCF modal parameter estimation method}\label{sec:POLYMAX-DERIVATION}
\section{The right-matrix rational fractional model}
The polyreference least-squares complex frequency domain method employs a right matrix fractional model to fit MIMO Frequency Response Function measurements into a set of rational polynomial transfer functions:
\begin{equation}\label{eq:plscf_poly}
    [H(\omega)] = [N(\omega)][D(\omega)]^{-1}
\end{equation}
Such that $H(\omega) \in\mathbb{C}^{N_{outputs} \times N_{inputs}} $ is the FRF matrix, where $D(\omega)\in \mathbb{C}^{N_{inputs} \times N_{inputs}}$, is the denominator matrix polynomial, and $N(\omega) \in \mathbb{C}^{N_{outputs} \times N_{inputs}}$, is the numerator matrix polynomial. The rows corresponding to each output $o$ in the FRF matrix can be represented as such:

\begin{equation}\label{eq:plscf_frf_row}
    \left \langle H_o(\omega) \right \rangle = \left \langle N_o(\omega) \right \rangle [D(\omega)]^{-1}
\end{equation}

The row vector numerator polynomial for the $o^{th}$ output, and the denominator matrix polynomial are defined in terms of a polynomial basis function, $\Omega(\omega)$, and their respective polynomial coefficients, $\beta$ and $\alpha$ as such:

\begin{equation}\label{eq:num_poly}
    \left \langle N_o(\omega) \right \rangle  = \sum_{r=1}^{p} \Omega_r(\omega) \left \langle \beta_{or}(\omega) \right \rangle
\end{equation}
\begin{equation}\label{eq:den_poly}
    [D(\omega)] = \sum_{r=1}^{p} \Omega_r(\omega)[\alpha_r]
\end{equation}
With the polynomial basis function $\Omega_r(\omega) = e^{j\omega\Delta tr}$. Although not initially obvious as polynomials with conventional form $p(x) = c_0 +c_1x +c_2x^2 +\dots + c_nx^n$, the basis functions are expressed in the $s$-domain where $s = e^{j\omega\Delta t}$. The polynomial coefficients, $\alpha_r \in \mathbb{R}^{N_{inputs}\times N_{inputs}}$ and $\beta_{or} \in \mathbb{R}^{1 \times N_{inputs}}$, are assembled into matrix form:

\begin{flalign}\label{eq:beta}
    \beta_o = \begin{pmatrix}
        \beta_{o0}\\
        \beta_{o1}\\
        \beta_{o2}\\
        \dots\\
        \beta_{op}
    \end{pmatrix}
    \in \mathbb{R}^{(p+1)\times N_{inputs}} &&
\end{flalign}


\begin{flalign}\label{eq:alpha}
    \alpha = \begin{pmatrix}
        \alpha_0\\
        \alpha_1\\
        \alpha_2\\
        \dots\\
        \alpha_p
    \end{pmatrix}
    \in \mathbb{R}^{N_{inputs}*(p+1)\times N_{inputs}} &&
\end{flalign}

\begin{flalign}\label{eq:theta}
    \theta = \begin{pmatrix}
        \beta_0\\
        \beta_1\\
        \beta_2\\
        \dots\\
        \beta_{N_{o}}\\
        \alpha
    \end{pmatrix} \in \mathbb{R}^{(N_{outputs}+N_{inputs})(p+1)\times N_{inputs}} &&
\end{flalign}
\section{Minimizing the sum of the squared residuals}
The collection of both sets of coefficients into one variable $\theta$, makes performing the least squares problem simpler, in a sense, as it becomes the one unknown in this least squares model. As typical in any fitting method, one must minimize the error between the model and the real or measured value. The nonlinear least-squares error for:\\
Measured FRF:\hspace{30pt} $\hat{H}_o(\omega_k)$\\
Model FRF: \hspace{42pt} $H_o(\omega_k)$\\
is weighted such that:
\begin{equation}\label{eq:nls_error}
    \epsilon_{o}^{NLS} (\theta,\omega_k) = w_o(\omega_k)(H_o(\omega_k)-\hat{H}_o(\omega_k))
\end{equation}
Where $\epsilon_{o}^{NLS} \in \mathbb{C}^{1\times N_{inputs}}$,   $w_o(\omega_k)$ is a scalar weighing function which captures the variation and deviation between multiple inputs on the same measurement point, and $\forall k = 0, 1, 2, \dots , N_{frequency}$. Said weighing function is typically denoted by
\begin{equation}\label{eq:weighing_fn}
    w_o(\omega_k) = \frac{1}{\sqrt{\mathbf{var}[H_o(\omega_k)]}}
\end{equation}
(See [reference for weighted linear regressions] for more information on weighted least squares.)\\
One can then define the nonlinear cost function as the sum of the error "squared", (hermitian inner product), over the data points, in this case, spectral lines and outputs;
\begin{equation} \label{eq:nls_cost_fn}
    l^{NLS}(\theta) = \sum_{o=1}^{N_{out}}\sum_{k=1}^{N_f}\mathbf{tr}\{(\epsilon_{o}^{NLS} (\theta,\omega_k))^H \epsilon_{o}^{NLS} (\theta,\omega_k)  \}
\end{equation}
In this equation, $\mathbf{tr}\{\bullet\}$ denotes the trace of a matrix, also known as the sum of diagonal elements, and $\bullet^H$ denotes the Hermitian (conjugate) transpose. The trace operator is used as the trace of a product of 2 matrices $\mathbf{A}\in \mathbb{C}^{m\times n}$ and $\mathbf{B}\in \mathbb{C}^{m\times n}$ will equal the sum of each individual element in $\mathbf{A}$ with the individual elements of $\mathbf{B}$. This provides a sum of all square residuals/errors in the cost function.
\begin{equation}\label{eq:tr_cyc_prop}
    \mathbf{tr}\{\mathbf{A}^H\mathbf{B}\} = \mathbf{tr}\{\mathbf{A}\mathbf{B}^H\} = \mathbf{tr}\{\mathbf{B}^H\mathbf{A}\} =\mathbf{tr}\{\mathbf{B}\mathbf{A}^H\} = \sum_{i=1}^{m}\sum_{j=1}^{n}a_{ij}b_{ij}
\end{equation}
\section{Linearizing the error}
One can then obtain the polynomial coefficients through minimizing the cost function in \ref{eq:nls_cost_fn}, by setting the derivative $\frac{\partial l^{NLS}}{\partial \theta}$ equal to zero, however a nonlinear cost function will yield nonlinear derivative equations, (typically called normal equations in linear regression). A subsequent linearization of the cost function can approximate (suboptimally) the least squares problem, this is achieved through right multiplying the cost function with denominator polynomial $\mathbf{D}$. This gives a linear error:
\begin{equation}\label{eq:lin_error}
    \begin{aligned}
        \epsilon^{LS}_o(\omega_k,\theta)  & = w_o(\omega_k) (N_o(\omega_k,\beta_o)-\hat{H}_o(\omega_k)D(\omega_k,\alpha))\\
        & = w_o(\omega_k)  \sum_{r=0}^{p}(\Omega_r(\omega_k)\beta_{or}-\Omega_r(\omega_k)\hat{H}_o(\omega_k)\alpha_r)
    \end{aligned}
\end{equation}
Stacking the error in terms for all spectral lines in one matrix $E^{LS}_o(\theta) \in \mathbb{C}^{N_f \times N_in}$:
\begin{equation}\label{eq:stacked_error}
    E^{LS}_o(\theta) = \begin{pmatrix}
        \epsilon^{LS}_o(\omega_1,\theta)\\
        \epsilon^{LS}_o(\omega_2,\theta)\\
        \epsilon^{LS}_o(\omega_3,\theta)\\
        \dots\\
        \epsilon^{LS}_o(\omega_{N_f},\theta)
    \end{pmatrix}
    = \begin{pmatrix}
        X_o & Y_o
    \end{pmatrix}
    \begin{pmatrix}
        \beta_o\\
        \alpha
    \end{pmatrix}
\end{equation}

Here, new variables $\mathbf{X}$ and $\mathbf{Y}$ are introduced:
\begin{equation}\label{eq:X}
    X_o = \begin{pmatrix}
        w_{o}(\omega_{1})\Bigl(\Omega_{0}(\omega_{1}) + \Omega_{1}(\omega_{1}) \dots \Omega_{p}(\omega_{1})\Bigr) \\
        w_{o}(\omega_{2})\Bigl(\Omega_{0}(\omega_{2}) + \Omega_{1}(\omega_{2}) \dots \Omega_{p}(\omega_{2})\Bigr) \\
        \vdots \\
        w_{o}(\omega_{N_{f}})\Bigl(\Omega_{0}(\omega_{N_{f}}) + \Omega_{1}(\omega_{N_{f}}) \dots \Omega_{p}(\omega_{N_{f}})\Bigr)
        \end{pmatrix}
        \in \mathbb{C}^{N_f \times (p+1)}
\end{equation}
\begin{equation}\label{eq:Y}
    Y_o = \begin{pmatrix}
        -w_{o}(\omega_{1})\Bigl(\Omega_{0}(\omega_{1}) + \Omega_{1}(\omega_{1}) \dots \Omega_{p}(\omega_{1})\Bigr) \otimes \hat{H}_o(\omega_1) \\ 
        -w_{o}(\omega_{2})\Bigl(\Omega_{0}(\omega_{2}) + \Omega_{1}(\omega_{2}) \dots \Omega_{p}(\omega_{2})\Bigr)\otimes \hat{H}_o(\omega_2)  \\
        \vdots \\
        -w_{o}(\omega_{N_{f}})\Bigl(\Omega_{0}(\omega_{N_{f}}) + \Omega_{1}(\omega_{N_{f}}) \dots \Omega_{p}(\omega_{N_{f}})\Bigr) \otimes \hat{H}_o(\omega_{N_f}) 
        \end{pmatrix}
        \in \mathbb{C}^{N_f \times N_{in}(p+1)}
\end{equation}
Where $\otimes$ is the Kronecker product. In these equations, $\mathbf{X}$ is used to capture the frequency content of the least squares problem, and $\mathbf{Y}$ is used to capture both the frequency content and the measured response data. Using these matrices, one can reconstruct the nonlinear cost function into one that is linear:
\begin{equation}
    \begin{aligned}\label{eq:lin_cost_fn}
        l^{LS}(\theta) & = \sum_{o=1}^{N_{out}}\sum_{k=1}^{N_{f}}\mathbf{tr}\{(\epsilon^{LS}_o(\omega_k,\theta))^H \epsilon^{LS}_o(\omega_k,\theta) \}\\
        & = \sum_{o=1}^{N_{out}}\mathbf{tr}\biggl\{(E^{LS}_o(\theta))^H E^{LS}_o(\theta) \biggr\}\\
        & = \sum_{o=1}^{N_{out}}\mathbf{tr} \Biggl\{\begin{pmatrix}
            \beta^{T}_o & \alpha^{T}
        \end{pmatrix}
        \begin{pmatrix}
            X^{H}_o\\Y^{H}_o
        \end{pmatrix}
        \begin{pmatrix}
            X_o&Y_o
        \end{pmatrix}
        \begin{pmatrix}
            \beta_o \\ \alpha
        \end{pmatrix}\Biggr\}
    \end{aligned}
\end{equation}
If one defines a \textit{Jacobian} matrix $\mathbf{J}\in \mathbb{C}^{N_f N_{out}\times (N_{in}+N_out)(p+1)}$ for the problem as such:
\begin{equation}\label{eq:jacobian}
\mathbf{J} = \begin{pmatrix}
    X_1 & 0 & \dots & 0 & Y_1\\
    0 & X_2 & \dots & 0 & Y_2\\
    \dots & \dots & \dots & \dots & \dots\\
    0 & 0 & 0 & X_{N_{out}} & Y_{N_{out}}
\end{pmatrix}
\end{equation}
The cost function can be represented as:
\begin{equation}\label{eq:cost_fn_simp}
    l^{LS}(\theta) = \mathbf{tr}\{\theta^T \mathbf{J}^H \mathbf{J} \theta \}
\end{equation}
To obtain real values of $\theta$, one must place a constraint on the cost function such that:
\begin{equation}\label{eq:cost_fn_simp_real}
    l^{LS}(\theta) = \mathbf{tr}\{\theta^T \mathit{Re}(\mathbf{J}^H \mathbf{J}) \theta \}
\end{equation}
Where the Gramian matrix of $\mathbf{J}$ can be represented in terms a set of variables, $\mathbf{R}$,$\mathbf{S}$ and $\mathbf{T}$:
\begin{equation}
    \mathit{Re}(\mathbf{J}^H\mathbf{J}) = \begin{pmatrix}
        R_1 & 0 & \dots & 0 & S_1\\
        0& 0 & \dots & 0 & S_2\\
        \dots & \dots & \dots & \dots & \dots\\
        0 & 0 & \dots & R_{N_{out}} & S_{N_{out}}\\
        S^{T}_1 & S^{T}_2 &\dots &S^{T}_{N_{out}} &\sum_{o=1}^{N_{out}}T_o
    \end{pmatrix}
\end{equation}
in which:
\begin{equation}\label{eq:r-tensor}
    R_o = \mathit{Re}(X^{H}_o X_o)
\end{equation}
\begin{equation}\label{eq:s-tensor}
    S_o =\mathit{Re}(X^{H}_o Y_o)
\end{equation}
\begin{equation}\label{eq:t-tensor}
    T_o = \mathit{Re}(Y^{H}_o Y_o)
\end{equation}

\section{The \emph{Normal Equations}, and extracting the modal parameters}
The cost function can then be minimized in terms of $\alpha$ and $\beta$ to find the best least-squares fit:
\begin{equation}
    \begin{aligned}
        \frac{\partial l^{LS}(\theta)}{\partial\beta_o} = 2(R_o \beta_o +S_o\alpha) &= 0 \\
         & \forall O = 1,2,\dots,N_{out}   
    \end{aligned}
\end{equation}
\begin{equation}
    \frac{\partial l^{LS}(\theta)}{\partial\alpha} = 2\sum_{o=1}^{N_{out}}(S^{T}_o\beta_o +T_o \alpha)
\end{equation}
Giving normal equations of this least squares problem in terms of the wanted polynomial coefficients, one can also assemble those normal equations into 1 equation:
\begin{equation}
    \frac{\partial l^{LS}(\theta)}{\partial\theta} = 2\mathit{Re}(\mathbf{J}^H\mathbf{J})\theta = 0
\end{equation}
The denominator coefficients $\alpha$ are used to obtain the poles and the modal participation factors, which are sufficient information for the constructing a stabilization diagram. Hence, one can further reduce the normal equations by setting:
\begin{equation}
    \beta_o = R^{-1}_oS_o\alpha
\end{equation}
This yields the reduced normal equation:
\begin{equation}\label{eq:reduced_normal_eqn}
    \begin{aligned}
        \Biggl\{2\sum_{o=1}^{N_{out}}(T_o - S^{T}_oR^{-1}_oS_o)\Biggr\}\alpha &= 0\\
        \mathbf{M}\alpha & = 0 
    \end{aligned}
\end{equation}
For a non-trivial solution to the normal equation, a constraint is set on $\alpha$, where:
\begin{equation}
    \alpha_p = \mathbf{I}_{N_{in}}
\end{equation}
The rest of the denominator coefficients are then found using:
\begin{equation}
    \mathbf{M}(1:N_{in}*p,1:N_{in}*p)\begin{pmatrix}
        \alpha_0\\ \alpha_1 \\ \alpha_2 \\ \dots \\ \alpha_{p-1}
    \end{pmatrix}
    = \mathbf{M}(1:N_{in}*p,N_{in}*p + 1 : N_{in}*(p+1))
\end{equation}
The least-squares estimate for $\alpha$ is then:
\begin{equation*}
    \hat{\alpha}_{LS} = \begin{Bmatrix}
        \alpha\\ \mathbf{I}_{N_{in}}
    \end{Bmatrix}
\end{equation*}

This makes the denominator polynomial $\mathbf{D}$ a \emph{monic} polynomial. Based on the fundamental definition of system poles, which is the points at which the system's response is "infinite", one can say that for an arbitrary rational polynomial $p(x)$:
\begin{equation}
    p(x) = \frac{a_0 +a_1x+a_2x^2+\dots+a_nx^n}{b_0 +b_1x+b_2x^2+\dots+b_nx^n}
\end{equation}
To achieve said "infinite" value, one must find values of $x$ that represent the roots of the denominator polynomial. In the pLSCF model, one can exploit the monic property of the denominator polynomial. Frobenius companion matrices are square matrices that represent monic polynomials, given one has a monic polynomial
\begin{equation*}
    p(x) = x^n +a_{n-1}x^{n-1}+a_{n-2}x^{n-2}+\dots +a_1x+a_0
\end{equation*}
The companion matrix of said polynomial is defined as:
\begin{equation}
    C(p) = \begin{bmatrix}
        0 & 0 & \dots &0& -a_0\\
        1 & 0 & \dots &0& -a_1\\
        0 & 1 & \dots &0& -a_2\\
        \vdots & \vdots & \ddots & \vdots&\vdots\\
        0 & 0 & \dots & 1 & -a_{n-1}
    \end{bmatrix}
\end{equation}
A property of the companion matrix $C(p)\in \mathbb{R}^{n\times n}$ is that its eigenvalues $\lambda_1, \lambda_2, \dots, \lambda_n$ are the roots of $p(x)$, where $p(\lambda_i) = 0, \forall i =1,2,\dots,n$. This property aids in finding the system poles, after constructing the companion matrix for the denominator polynomial, an Eigendecomposition of the matrix yields the discrete time poles as the eigenvalues, and the corresponding eigenvectors are the modal participation factors:
\begin{equation}
    C(\mathbf{D}) = \begin{pmatrix}
        0 & I &\dots & 0 & 0 \\
        0 & 0 &I & \dots & 0 \\
        \dots & \dots & \dots & \dots & \dots \\
        0 & 0 &\dots & 0 & I \\
        -\alpha_0^T & -\alpha_1^T &-\alpha_2^T& \dots -\alpha_{p-2}^T & -\alpha_{p-1}^T
    \end{pmatrix}
\end{equation}
\begin{equation}
    C(\mathbf{D})[\mathbf{L}] = \Lambda[\mathbf{L}]
\end{equation}
Where the matrix $[\mathbf{L}]$ is the Eigenmatrix (matrix with columns as eigenvectors), representing the modal participation factors of each mode, and the matrix $\Lambda$ contains the discrete time poles on its diagonal elements. The transpose of a companion matrix is used for pLSCF, this however does not affect the numerical value of the poles or participation factors [reference proof]. A $p$-order right matrix-fraction polynomial estimation should yield $pN_{in}$ number of poles.

\section{Finding the modeshapes using the Least-Squares Frequency Domain (LSFD) method}

In practice, modeshapes are estimated using the Least Squares Frequency Domain algorithm, which is a linear regression designed to find the modal residues, a product of the modeshapes and the participation factors, and the upper and lower residuals, values which account for non-ideal vibratory behaviour. The LSFD fits a new model to measured FRF data as shown in Equation \ref{eq:lsfd-model}.

\begin{equation}\label{eq:lsfd-model}
    H(s) = \sum_{m=1}^{N_{modes}}\left( \frac{\Psi_m\cdot L_{m}^{T}}{s - \lambda_m}+\frac{\Psi_{m}^{*}\cdot L_{m}^{H}}{s - \lambda_{m}^{*}} \right) +\frac{\mathbf{LR}}{s^2} + \mathbf{UR}
\end{equation}
Where $\Psi_m$ is the modeshape vector for the mode $m$, $L_{m}$ is the mode's participation factor, $\mathbf{LR}$ and $\mathbf{UR}$ are the upper and lower residuals, and $s = j\omega$. By taking the Modal Residues for the $i^{th}$ mode $[Res_i] = \Psi_m\cdot L_{m}^{T}$, they can be found, alongside the Upper and Lower residuals, in one single linear least squares step. By taking the Jacobian matrix $J$ to be
\begin{equation}
    J = \begin{bmatrix}
        \frac{1}{j\omega_2 - \lambda_1} + \frac{1}{j\omega_2 - \lambda_1^*} & \frac{1}{j\omega_2 - \lambda_2} + \frac{1}{j\omega_2 - \lambda_2^*} & \dots & \frac{1}{j\omega_2 - \lambda_{N_{m}}} + \frac{1}{j\omega_2 - \lambda_{N_{m}}^*} & \frac{-1}{\omega_2^2} & 1 \\

        \frac{1}{j\omega_3 - \lambda_1} + \frac{1}{j\omega_3 - \lambda_1^*} & \frac{1}{j\omega_3 - \lambda_2} + \frac{1}{j\omega_3 - \lambda_2^*} & \dots & \frac{1}{j\omega_3 - \lambda_{N_{m}}} + \frac{1}{j\omega_3 - \lambda_{N_{m}}^*} & \frac{-1}{\omega_2^2} & 1 \\

        \dots & \dots & \dots & \vdots  & \dots &\vdots \\

        \frac{1}{j\omega_{N_{f}} - \lambda_1} + \frac{1}{j\omega_{N_{f}} - \lambda_1^*} & \frac{1}{j\omega_{N_{f}} - \lambda_2} + \frac{1}{j\omega_{N_{f}} - \lambda_2^*} & \dots & \frac{1}{j\omega_{N_{f}} - \lambda_{N_{m}}} + \frac{1}{j\omega_{N_{f}} - \lambda_{N_{m}}^*} & \frac{-1}{\omega_{N_{f}}^2} & 1 \\
    \end{bmatrix}
\end{equation}

The residues and the upper and lower residuals can be estimated as:

\begin{equation}
    \begin{bmatrix}
        Re[Res]_{1o}\\
        \dots\\
        Re[Res]_{N_mo}\\
        Im[Res]_{1o}\\
        \dots\\
        Im[Res]_{N_mo}\\
        [LR]_o\\
        [UR]_o
    \end{bmatrix}
    = Re(J^H J)^{-1} Re(J^H H_o)
    \forall O = 1,2,3,\dots , N_{out}
\end{equation}

With no knowledge of the modal participation factors, the modeshape can only be estimated using a singular value decomposition (SVD) of the residue matrix.
\begin{equation}
    [Res_i] = \mathbf{U\Sigma V^T}
\end{equation}
Where $\mathbf{\Sigma}$ is a diagonal matrix containing the singular values of the matrix, $\mathbf{U}$ is a matrix containing left singular vectors, and $\mathbf{V}$ is a matrix containing right singular vectors. This process can yield accurate modeshapes only if the residue matrix is a rank-1, which mat not always be the case. Knowledge of the modal participation factors, which the pLSCF algorithm provides, removes this limitation by allowing a vector to be calculated using the properties of the outer product and rearranging. The use of the outer product provides quicker computation, as matrix-vector multiplication is generally considered quicker than the SVD, this is particularly important for the frequently encountered, high number of output systems.
\begin{equation}
    \mathbf{\Psi}_i = \frac{[Res_i] \mathbf{L}_i}{\mathbf{L}_i^H \mathbf{L}_i}
\end{equation}




% This part should be in actual report: When fitting theoretical models to experimental data, the difficulty does not typically lie in the mathematical framework enabling the modelling process. Instead, the challenge is in constructing a model that provides physically meaningful insights. Given that the goal of many modal parameter estimation methods is to find a set of poles which As apparent, it is straightforward to find the polynomial coefficients using measured data. 

\pagebreak
\addcontentsline{toc}{chapter}{References}
\bibliographystyle{IEEEtran}
\bibliography{references}

\end{document}