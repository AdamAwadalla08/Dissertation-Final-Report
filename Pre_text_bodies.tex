\chapter*{Acknowledgements}
\addcontentsline{toc}{chapter}{Acknowledgements}
I would like to express my deepest gratitudes to my primary supervisor, Dr. Tim Rogers, and my secondary supervisors, Dr. Max Champneys and Dr. Brandon O'Connel, for all of their unwavering support, guidance, and understanding throughout this project. Thanks also to my partner, Katie,  without whom I would not have submitted this work. Finally, my Father, for supporting me, his financial burden.

To everyone mentioned here, and to all those who have contributed—whether through discussion, critique, or moral support—I offer my heartfelt thanks.
\clearpage

\chapter*{Abstract}
\addcontentsline{toc}{chapter}{Abstract}
Structural dynamics relies on modal analysis to characterise vibratory behaviour in engineering systems, yet commercial software obscures underlying algorithms and imposes high licensing costs. This work aims to deliver a transparent, open-source alternative by implementing the polyreference least-squares complex frequency-domain estimator (pLSCF, PolyMAX) and a least-squares frequency-domain (LSFD) mode-shape routine in Python. Methods include a NumPy/SciPy - based architecture, rigorous unit- and integration-testing, PEP-8 documentation, and model-order-reduction via matrix truncation and singular-value decomposition to accelerate computation. Results on a ten-degree-of-freedom benchmark show natural frequencies and damping ratios within $9 \times 10^{-5} \%$  of analytical values; mode shapes reach modal-assurance-criterion scores $\geq$ 0.997. Computational time is reduced by up to two orders of magnitude relative to brute-force fitting, and accuracy is retained under 25 dB signal-to-noise conditions after peak-segmentation pre-processing. The study concludes that the presented library supplies a robust, high-performance, and freely accessible alternative to proprietary tools while providing a foundation for future enhancements such as MAC-based stability criteria, PolyMAX Plus noise handling, and an interactive graphical interface.
\clearpage

